\documentclass[letter]{article}

\usepackage{amsmath}
\usepackage{graphicx}
\usepackage{geometry}
\usepackage{braket}
\geometry{
  letterpaper,
  left=1in,
  right=1in,
  bottom=1in,
  top=1in}
\setlength\parindent{0pt}

\begin{document}
\textbf{\Large{Nuclear Engineering 101: Midterm 2 Study Guide}} \\
Joshua Rehak
\vspace{12pt}
%\cite[pp. 45]{krane}
%\cite[Lec 24]{lecture}

\section*{Nuclear Electromagnetic Moments}

Ref: \cite[pp. 71-75]{krane},\cite[Lec 10]{lecture}

A distribution of electric charge and/or current will create an
electric potential. This can be described by the ``multipole
moment''. \\

Note the values of $L$ are the \textit{order of the moment}, not
angular momentum.

\subsection*{Electric multipoles}

\begin{itemize}
\item Monopole moment: $L=0$, this is just the spherical nucleus, the electric
  potential looks like $V = \frac{Q}{R}$. 
\item Dipole moment: $L=1$. The electric dipole is just linear
  translation of the nucleus. $V=0$, see below.
\item Quadrupole moment: $L=2$, the electric potential is from the
  nucleus squishing in one direction and then the other. The amount of
  quadrupole moment can be calculated:
  \begin{equation*}
    eQ=e\int\psi^*(3z^2-r^2)\psi~d^3\vec{r}
  \end{equation*}
  If the nucleus is spherical,
  $\braket{z}^2=\braket{x}^2=\braket{y}^2$, so $eQ=0$. The higher the
  value of $eQ$ the more deformed the nucleus is.
\end{itemize}

All electric moments have parity determined by $(-1)^L$. If you want
to do some kind of electromagnetic operation on a state ($\psi$) with
some operator $O$, then you have to evaluate $\int\psi^*O\psi~dv$. It
doesn't matter what the parity of $\psi$ is because they will always
multiply (even times even or odd times odd are both always even). If
$O$ has negative parity, the whole thing is an odd function and the
integral goes to 0. Therefore, to keep things based in reality, all
 moments with odd angular momentum must not be a thing because they
 have negative parity
($L=1,3,5$ etc). This is why the dipole doesn't happen.

\subsection*{Magnetic multipoles}

\begin{itemize}
\item Monopole moment: $L=0$ doesn't happen, as far as we know.
\item Dipole moment: $L=1$, nucleons and nuclei have dipole magnetic
  moments due to their angular momentum (they are charges moving in a
  loop, therefore magnetic field). 
  \begin{equation*}
    \mu = \frac{e\hbar}{2m}l=g_ll\mu_n
  \end{equation*}
Here $l$ \textbf{is} the angular momentum, $g_l$ is a constant (1 for
protons, 0 for neutrons) and $\mu_n$ is the ``nuclear magneton'' (just
a constant).
\item Quadrupole moment: $L=3$ doesn't happen.
\end{itemize}

Just like before, all the moments with odd parity have to not
exist. For magnetic moments, the parity is determined by $(-1)^{L+1}$,
so all the \textit{even} order moments can't exist.

\section*{Shell Model}

\subsection*{Evidence}
\begin{itemize}
\item Ionization energies ``jump'' at ``magic numbers'' like 2, 10,
  18, 36, 54, 86. Something about these make the nucleus more tightly
  bound, this wouldn't be seen in a liquid
  drop ~\cite[Lec. 12]{lecture}.
\item $\alpha$-decay: see a big jump in the $\alpha$-decay of Radon
  after $N=128$. This is because the daughter has $N=126$ (magic
  number) ~\cite[Lec. 12]{lecture}.
\end{itemize}

\subsection*{Shells}
\begin{itemize}
\item You can use the 3D square well to get a pretty good
  approximation of what we see, or a parabolic ($1/r^2$) one to get
  better. But the simple harmonic oscillator gives the best
  approximation but it's still wrong. ~\cite[Lec. 12]{lecture}.
\item Spin-orbit interaction: due to the interaction of the spin and
  the angular momentum ($\vec{l}\bullet\vec{s}$), you can get two
  different values of $j$, $l \pm \frac{1}{2}$. Now two different
  nucleons will see a \textit{different} potential, so it splits all
  the states in two. \cite[Lec. 13-16]{lecture},
  \cite[pp. 123-125]{krane}.
\item The size of the split gets bigger with increasing $l$. The
  potential is negative so the $J=l+\frac{1}{2}$ states will occur at
  \textit{lower} energies. High spin states that enter a different lower shell are called \textit{the
  intruder}. \cite[Lec. 13-16]{lecture}
\item Evidence: Magnetic dipole $\mu=\mu_N(g_ll_z + g_ss_z)/\hbar$
  should be different for the different spins in the same $l$
  level. Observations show this is true. \cite[Lec. 13-16]{lecture}.
\item Works well for \textbf{spherical} nuclei and ones that are
  \textbf{close to magic numbers}.
\end{itemize}

\subsection*{Independent Particle Model}
\begin{itemize}
\item All shells are completely full or empty except for a single
  particle in the lowest energy of an otherwise empty shell. The
  nucleus $J^\pi$ depends only on that last
  nucleon. \cite[Lec. 13-16]{lecture}.
\item The total angular momentum of the nuclei $J$ is equal to the
  $J$ value of the shell with the single particle in it. The parity
  is equal to the parity of the shell with the single particle in it,
  $(-1)^l$. For example, if there is a single particle in the
  $1p_{3/2}$ state, then $J^\pi=(\frac{3}{2})^{-}$ because
  $j=\frac{3}{2}$ and $l=1$. \cite[Lec. 13-16]{lecture}.
\item The value $J^\pi$ is referred to as the spin-parity \textbf{or}
  the total angular momentum of the nucleus. Both of those
  things mean the same thing \textit{when you're talking about the
    nucleus.} For a nucleon, spin is the intrinsic angular momentum;
  the nucleus doesn't have any \textbf{intrinsic} angular momentum,
  only angular momentum due to it's components. We call it spin anyway
  when referring to the nucleus just to be confusing.
\item The independent nucleon always gives the total angular momentum
  of the shell to the nucleus when in the \textbf{ground state.}
\item When you are in an \textbf{excited state}, you can line up the
  nucleons in more interesting way. If you have three nucleons in the
  $f_{7/2}$ shell and we're in an excited state, we use the notation
  $(f_{7/2)})^3$. You can add up their angular momentum any way you
  want, remember each can have angular momentum from $-j$ to $+j$. So
  these can have $m=\pm\frac{1}{2}, \pm\frac{3}{2}, \pm\frac{5}{2},
  \pm\frac{7}{2}$. They can't have the \textbf{same} value of m, but
  they can all add to get, for example
  $\frac{7}{2}+\frac{5}{2}+\frac{3}{2}=\frac{15}{2}$. \cite[pp. 149-151]{krane}
\item In excited states, the single nucleons can also jump around to
  other states, changing the $J^\pi$ of the nucleus. \cite[Lec. 13-16]{lecture}
\end{itemize}


\bibliographystyle{unsrt}
\bibliography{NE101}
\end{document}
